% TODO: Pictures

\documentclass[12pt, a4paper]{article}

\usepackage{preamble}

% TODO: Decide who focuses on which parts. There will of course be overlap and
% we will all review and discuss the different sections.

\title{Models of Associative Memory}

\author{Wenting Jin, Lucas Arnström \& Robin Eklind}

\begin{document}

\maketitle

\tableofcontents

\clearpage

% TODO: Add abstract?

% === [ Introduction ] =========================================================

\section{Introduction}

\subsection{Associative Memory}

Definition of associative memory.

Distinction between auto-associative and hetero-associative memory.

% === [ Models of Associative Memory ] =========================================

\section{Models of Associative Memory}

\subsection{Hopfield Networks}

% Wenting.

In paper from J.J.Hopfield [1982], the drawbacks of Perceptron were addressed through its intractable back-coupling, lack of abstraction properties and requirement of synchrony. Information storage was improved with help of Nonlinearity, and emergent computational properties were obtained from simple properties of many cells rather than complex circuitry(which is a result of linear associativity). The input-output relationship of nonlinear computation and binary threshold units were introduced.

Collective behaviors of the model was studied and resulted with the following findings: a few stable states was resulted from most of the initial state space, properties necessary for a physical content-addressable memory were not dependent on the symmetry of the connectivity matrix $T_{ij}$. Statement supported by findings from experiments is that "about 0.15N(memory storage bits) states can be simultaneously remembered before error in recall is severe". Case with arbitrary starting state was studied and results of memories near to the starting state was highly produced. The nominally assigned memories which were called "attractors" dominates the phase flow whereas the flow is not entirely deterministic, which leads to a convergence to local optimum.

Case of consistent internal correlations in the memory was as well adressed, and Hebb synaspses was used and slightly modified to generate nonsymmetric term~$\delta T_{ij}$, which limitation of sequence of four states was addressed.

Mostly auto-associative. Also hetero-associative?

\subsection{Bidirectional Associative Memory}

Mostly hetero-associative. Also auto-associative.

\subsection{Boltzmann Machine}

% --- Lucas

The "Boltzmann Machine" (BM) is a form of "parallel constraint satisfaction network" \cite{ackley1985learning}. It is capable of learning the underlying constraints of a domain by only being shown examples of it. The BM is composed of units forming a complete graph where the connection between two units are symmetric, meaning that the weight on the connection is the same in either direction. No unit has a connection to itself. The units are binary, meaning that they can assume one of two states, on or off. The state of a unit is determined by a probabilistic function based on the states of the units neighbours. A strong connection (high weight value) between two units indicates that if either of these two units are active, the other one should probably be active as well. While a weak connection (low weight value) indicates that these should probably not be active at the same time.

The BM is notably similar to the Hopfield network in that it also defines a global energy state of the system, utilizing the same equation that determines the global energy value. Each global state can be identified by the energy of the system in that state. By forcing the values of the visible units to represent a training set the system attempts to find an energy configuration that is compatible with the given input. The resulting energy state can then be interpreted as to how well the given data fulfills the constraints of the domain. Thus by minimizing the energy the system learns an interpretation of the problem that increasingly satisfies the constraints of the domain.

The simplest way to minimize the energy of the system is to change each unit into a value that results in a lower energy state. The data needed to determine this change is locally accessible to each unit. If the sum of all values for a given units neighbour exceeds the threshold of that unit, the resulting state of the unit should be on. Otherwise it should be off. This is the usual algorithm for binary units.

Because of this deterministic algorithm it suffers from the usual weaknesses of gradient descent algorithms, namely it gets stuck in local minima if its initial state is close to one. In order to alleviate the algorithm of this problem noise is introduced in the training. This allows the network to "jump" out of these minima into configurations of higher energy. The algorithm used for noise introduction is a variation of the "Metropolis algorithm" \cite{metropolis1953equation} that was used to study thermodynamic systems. This modified version introduces a concept of temperature to the machine. The machine then tries to reach "thermal equilibrium" during training. Meaning that the machine is allowed to run repeatedly until the global energy of the system converges to a fixed state over a temperature that initially is high and lowered during the runtime of the system.

%TODO write more about this temperature thing
%TODO write more about the probabilistic features of the BM as well as the boltzmann distribution that gives the system its name.

Training is conducted in two phases, the first is called the "positive" phase were the visible units of the machine are set to the values of the training set. The next phase is called the "negative" phase. During this phase the machine is allowed to run freely, independent of the training set.

% --- NOTES
%The difference between the BM and the Hopfield network is mainly that the nodes (or units as they are referred to in the original paper) of the Boltzmann Machine are stochastic by nature.
%The BM can be used for constraint satisfaction problems that involve a large amount of weak constraints.



%\subsection{Hamming networks}

%foo

% TODO: Decide whether to include Hamming networks or not.

\subsection{Memory Resistor}

%Robin.

Memory resistors, memsistors for short, are passive circuits which change their resistance as current flows through them, and maintain their resistance in between use. In other words, memsistors remember past flows of current (i.e. store state), and control resistance based on this knowledge (i.e. stored state). Several similarities have been identified bewteen the properties of memsistors and synapses, which make them interesting candidates for associative memory models.

\cite{memsistor} \cite{ahah}

% === [ Current Capabilities ] =================================================

\section{Current Capabilities}

Definitions of capacity.

\begin{itemize}
\item Absolute capacity.
\item Relative capacity.
\item Capacity of associative memory
\item Relative capacity of recalling process
\end{itemize}

\subsection{Hopfield Network}

%Wenting.

1982, $0.15n$ (capacity of associative memory)

% Above 0.15n releases the constraint on symmetries according to Olle.

1985, proven $ \frac{n}{2\log{n}} $ (absolute capacity)

1985, $0.14n$ (relative capacity of recalling process)

1993, $ n ~= 0.4n $ (new result, absolute capacity)

\subsection{Bidirection Associative Memory}

% TODO: Check.

foo

\subsection{Boltzmann Machine}

%Lucas.

%Pure Boltzman does not scale, thus impractical.
%Restricted Boltzman, basis for deep learning today, using multiple layers.

\subsection{Memory Resistor}

%Robin.

Proof of concept in 2010, using 3 neurons and 2 synapses to achieve 1 associative memory formation.

% === [ Future Potential ] =====================================================

\section{Future Potential}

\subsection{Intelligent Systems}

Intelligence defined by ability to make predictions, not behaviour \cite{intelligence_is_prediction}.

\subsection{Energy Efficiency}

Network in its true sense, \textit{the net is doing all the work} \cite{net_doing_all_the_work}.

%Current computer architectures are designed around major bottlenecks, huge amounts of data has to be shuffled back and forth to perform computations. Reaching its limits; transistors now so small that they only allow a single electron to pass through (similar in size to the ion channels). At this scale, problems arise when transistors may allow electrons to pass through, when they shouldn't and wise versa; which leads to unpredictable behaviour (small bursts of ones when should be be all zero, and wise versa.)

%The brain, parallel, error correcting, memory efficient. (not doing a lot of unnecessary data shuffling?)

% === [ Conclusion ] ===========================================================

\section{Conclusion}

foo

% === [ Research Literature ] ==================================================

Preliminary list of references, cited to force inclusion within the bibliography.

% TODO: Add to this list throughout the project.

% Wenting
\cite{computational_abilities} \cite{optical_processing} \cite{capacity_of_nonmonotonic_model} \cite{stimulus_unitization_and_aging}

% Lucas
\cite{ackley1985learning} \cite{capacity_of_hopfield} \cite{high-order_neural_networks} \cite{neural_network_models_for_associative_memory} \cite{sparsely_encoded_associative_memory}

% Robin
\cite{memsistor} \cite{principle_of_neural_associative_memory} \cite{parallel_models_of_associative_memory} \cite{associative_memory_using_small-world_architecture} \cite{associatron} \cite{associative_search_network} \cite{deep_machine_learning} \cite{on_associative_memory} \cite{from_cell_to_cortex} \cite{ahah}

\bibliography{references}

\end{document}
