% +--- Anti-Hebbian and Hebbian
% |
% V

% ref: ref: http://knowm.org/report-from-the-navy-karles-invitational-on-neuro-electronics/
%
% Where each operation results in Anti-Hebbian or Hebbian learning. At the lowest level, Anti-Hebbian just means “move the synapse toward zero” and Hebbian means “move it away from zero”.

% ref: http://knowm.org/knowm/
%
% “Anti-Hebbian and Hebbian” in honour of Donald O. Hebb

% ref: https://www.youtube.com/watch?v=CFSrC7kjbJo ("Introduction to AHaH Computing, Alex Nugent, RIT, April 2015")
%
% > Hebbian (erase the path): Any modification to the synaptic weight that reduces the probability the synaptic state will remain upon subsequent measurement.
% >
% > Anti-Hebbian (select the path): Any modification to the synaptic weight that increases the probability the synaptic state will remain the same upon subsequent measurement.

% ref: https://www.youtube.com/watch?v=CFSrC7kjbJo ("Introduction to AHaH Computing, Alex Nugent, RIT, April 2015")
% >
% > Naturally they go towards zero. But if you give them a burst, you can make them positive or negative as you want. Read brings it a little closer together, then reward it.
