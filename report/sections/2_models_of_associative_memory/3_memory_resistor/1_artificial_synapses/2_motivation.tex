% +--- motivation behind AHaH nodes and new paradigm of computing.
% |
% V

% ref: http://knowm.org/knowm-api/
%
% Every modern computing system currently separates memory and processing. This works well for many tasks, but it fails for large-scale adaptive systems like brains or large ML models like neural networks. Indeed, there is no system in Nature outside of modern human digital computers that actually separates memory and processing, so it’s a wonder we have been able to do as much as we have.

% ref: http://knowm.org/
%
% > ## Modern computing
% >
% > Clear distinction between memory and processing.
% > Shuttle information back and forth.
% > Takes too long, requires too much energy.

% ref: https://www.youtube.com/watch?v=Tb2E-t11OH4 ("What is AHaH Computing?")
%
% > Large scale adaptive learning systems, like brains.
% >
% > Bring memory and processing closer together. Parallel computing.
% >
% > Reduce or eliminate the energy normally associated with computing those functions.

% ref: https://www.youtube.com/watch?v=R7HxFhVQVr4 ("Why AHaH computing?")
%
% > Current digital computing platforms are billions of times less power and space efficient than biology (brains) for synaptic operations.
% >
% > It's physically not possible to reach biological level efficiency without significant changes in both architecture and technology.
% >
% > "GPU's will save us." `No they won't, they just suck less`
% >
% > If you insist on separating memory and computing, you won't even get close to the capacity of biology.
% >
% > Traditional computing would go an inch while biology is able to circle the world.
