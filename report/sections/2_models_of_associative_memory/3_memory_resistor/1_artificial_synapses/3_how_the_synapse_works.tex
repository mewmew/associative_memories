% +--- Synapse consists of two memristors, competing.
% |
% V

% ref: https://www.youtube.com/watch?v=IVDRcV8XvlI ("What is an AHaH node?")
%
% > Two memristors competing with each other.
% >
% > A kT-bit (thermodynamic bit), a synapse consisting of two memristors.
% >
% > Look at the leaf of a plant. Energy dissipating system, constructed of many bifurcating channels. Every place where the energy flow splits, you have an AHaH node. You have two competing energy dissipating pathways.
% >
% > Nature is built of AHaH nodes.
% >
% > Understand how things self-organize.

% ref: http://knowm.org/knowm/
%
% A memristor is the electronic equivalent of an adaptive container!
%
% “Knowm's Synapse” or “Nature’s Transistor”. Two competing energy dissipation pathways.
%
% As a voltage (pressure) is applied to a memristor, its conductance will change.
%
% It is responsible for most self-organization on this planet. Nature is built of Knowms, including you.
%
% It is created when two energy dissipation pathways are competing for conduction resources. It appears to be at the heart of most self-organization.
%
% Knowm is built of a simple part repeated over and over again.
%
% When energy (for example water) flows through an adaptive container (for example dirt), the medium adapts or erodes in a particularly way that causes the energy to be dissipated faster. For example, the water erodes the ground and causes a channel to grow, which lowers the resistance to flow.

% ref: https://www.youtube.com/watch?v=NO9kmqr8NLk ("The Adaptive Power Solution")
%
% > AHaH circuit. An intrinsic adaptation mechanism. Energy dissipation pathways competing for conduction resources.
% >
% > Everywhere in nature. Competing energy dissipating pathways. Fractal.
% >
% > A memristor is effectively an adaptive energy dissipating pathway. It's like a riverbed. As you pass current through it, it will change its resistance, the riverbed will get bigger. Make memristors compete for energy dissipation.
% >
% > AHaH - Anti-Hebbian and Hebbian plasticity
% >
% > "Maximization of energy dissipation" - Swenson
% > "Maximization of currents" - Bejan
% > "Dissipation-driven adaptation of matter" - England
% > "Energy dissipation pathways competing for conduction resources" as a mechanism.

% ref: https://www.youtube.com/watch?v=CFSrC7kjbJo ("Introduction to AHaH Computing, Alex Nugent, RIT, April 2015")
%
% > A memristor for each pathway, and they are competing with each other so you need two per synapse. Think of the pair as a synapse itself, and the different in conductance between the two is the value of the synapse. If one is more conductive than the other it is positive, and vice versa, its negative.
