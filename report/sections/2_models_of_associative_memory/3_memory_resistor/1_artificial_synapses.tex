\subsubsection{Artificial Synapses}

% TODO: Continue from here.

Several similarities have been identified between the properties of memristors and synapses, which make them interesting candidates for associative memory models; as further described in section \ref{sec:current_capabilities_memory_resistor}.

% +--- New paradigm
% |
% V

% ref: http://knowm.org/
%
% > ## New paradigm of computing
% >
% > The act of accessing memory is the act of processing.

% +--- motivation behind AHaH nodes and new paradigm of computing.
% |
% V

% ref: http://knowm.org/knowm-api/
%
% Every modern computing system currently separates memory and processing. This works well for many tasks, but it fails for large-scale adaptive systems like brains or large ML models like neural networks. Indeed, there is no system in Nature outside of modern human digital computers that actually separates memory and processing, so it’s a wonder we have been able to do as much as we have.

% ref: http://knowm.org/
%
% > ## Modern computing
% >
% > Clear distinction between memory and processing.
% > Shuttle information back and forth.
% > Takes too long, requires too much energy.

% ref: https://www.youtube.com/watch?v=Tb2E-t11OH4 ("What is AHaH Computing?")
%
% > Large scale adaptive learning systems, like brains.
% >
% > Bring memory and processing closer together. Parallel computing.
% >
% > Reduce or eliminate the energy normally associated with computing those functions.

% ref: https://www.youtube.com/watch?v=R7HxFhVQVr4 ("Why AHaH computing?")
%
% > Current digital computing platforms are billions of times less power and space efficient than biology (brains) for synaptic operations.
% >
% > It's physically not possible to reach biological level efficiency without significant changes in both architecture and technology.
% >
% > "GPU's will save us." `No they won't, they just suck less`
% >
% > If you insist on separating memory and computing, you won't even get close to the capacity of biology.
% >
% > Traditional computing would go an inch while biology is able to circle the world.

% +--- Synapse consists of two memristors, competing.
% |
% V

% ref: https://www.youtube.com/watch?v=IVDRcV8XvlI ("What is an AHaH node?")
%
% > Two memristors competing with each other.
% >
% > A kT-bit (thermodynamic bit), a synapse consisting of two memristors.
% >
% > Look at the leaf of a plant. Energy dissipating system, constructed of many bifurcating channels. Every place where the energy flow splits, you have an AHaH node. You have two competing energy dissipating pathways.
% >
% > Nature is built of AHaH nodes.
% >
% > Understand how things self-organize.

% ref: http://knowm.org/knowm/
%
% A memristor is the electronic equivalent of an adaptive container!
%
% “Knowm's Synapse” or “Nature’s Transistor”. Two competing energy dissipation pathways.
%
% As a voltage (pressure) is applied to a memristor, its conductance will change.
%
% It is responsible for most self-organization on this planet. Nature is built of Knowms, including you.
%
% It is created when two energy dissipation pathways are competing for conduction resources. It appears to be at the heart of most self-organization.
%
% Knowm is built of a simple part repeated over and over again.
%
% When energy (for example water) flows through an adaptive container (for example dirt), the medium adapts or erodes in a particularly way that causes the energy to be dissipated faster. For example, the water erodes the ground and causes a channel to grow, which lowers the resistance to flow.

% +--- Evolve to solve problem
% |
% V

% ref: https://youtu.be/dgbooumJ4Tg?t=1692
%
% Won a nobel price. Eliol Prigoge.
% There is a natural tendency for complex systems to minimize the energy consumption

% +--- Mechanism for repair; inherent
% |
% V

% ref: https://www.youtube.com/watch?v=ZBJX6zzwnRI ("The Adaptive power problem.")
%
% > Noise is everywhere.
% >
% > Noise margin (analogue vs. digital).
% >
% > The signal gets corrupted by the noise.
% >
% > Memristors. Two meta-stable switches. Potential energy that has to be overcome to have a transition.
% >
% > We want: Low power + adaptation ("ability to change")
% > But: Parts will constantly break (because of noise), decay, volatility
% > Consequently: We need a mechanism of repair.
% >
% > The Adaptive Power Problem
% > Low Power + Adaptation = Parts break
% >
% > Intelligence -> Learning -> Adaptation

% ref: https://www.youtube.com/watch?v=NO9kmqr8NLk ("The Adaptive Power Solution")
%
% > Intrinsic mechanism of repair; inherent.
% >
% > What if constant adaptation *is* the mechanism of repair?
% >
% > What is the "essential nature" of adaptation?
% >
% > When nature minimizes its potential energy, it also solves our problem."; e.g. minimal surface with soap bubbles.
% >
% > To repair yourself is to be alive. Death is decay.
% >
% > Bejan (Construcal Law):
% > "For a finite-size system to persist in time (to live), it must evolve in such a way that it provides easier access to the imposed currents that flow through it."
% >
% > Swenson:
% > "A system will select the path or assembly of paths out of available paths that minimizes the potential or maximizes the entropy at the fastest rate given the constraints."
% >
% > England:
% > "Dissipation-driven adaptation of matter."
% >
% > E.g. The system will go with the flow, and it will maximize the flow.
% >
% > Maximize the dissipation of energy. The system will evolve in time towards that maximum, intrinsic repair.
% >
% > Have the system evolve itself to solve our problems.
% >
% > AHaH circuit. An intrinsic adaptation mechanism. Energy dissipation pathways competing for conduction resources.
% >
% > Everywhere in nature. Competing energy dissipating pathways. Fractal.
% >
% > A memristor is effectively an adaptive energy dissipating pathway. It's like a riverbed. As you pass current through it, it will change its resistance, the riverbed will get bigger. Make memristors compete for energy dissipation.
% >
% > AHaH - Anti-Hebbian and Hebbian plasticity
% >
% > "Maximization of energy dissipation" - Swenson
% > "Maximization of currents" - Bejan
% > "Dissipation-driven adaptation of matter" - England
% > "Energy dissipation pathways competing for conduction resources" as a mechanism.

% +---
% |
% V

% ref: https://www.youtube.com/watch?v=CFSrC7kjbJo ("Introduction to AHaH
%
% > XXX [ IMPORTANT ] XXX
% >
% > Synaptic access is processing is adaptation (memory and processing merged, d=0).
% >
% > XXX [/ IMPROTANT ] XXX

% ref: https://www.youtube.com/watch?v=CFSrC7kjbJo ("Introduction to AHaH Computing, Alex Nugent, RIT, April 2015")
%
% > Focuses on self-organizational building blocks.
% >
% > Points of bifurcation, branches.
% >
% > Energy dissipating fractal pattern.
% >
% > d (the distance) is zero in the brain, has tremendous effects on energy usage.
% >
% > A memristor for each pathway, and they are competing with each other so you need two per synapse. Think of the pair as a synapse itself, and the different in conductance between the two is the value of the synapse. If one is more conductive than the other it is positive, and vice versa, its negative.
% >
% > Naturally they go towards zero. But if you give them a burst, you can make them positive or negative as you want. Read brings it a little closer together, then reward it.
% >
% > Hebbian (erase the path): Any modification to the synaptic weight that reduces the probability the synaptic state will remain upon subsequent measurement.
% >
% > Anti-Hebbian (select the path): Any modification to the synaptic weight that increases the probability the synaptic state will remain the same upon subsequent measurement.
% >
% > XXX [ IMPORTANT ] XXX
% >
% > Synaptic access is processing is adaptation (memory and processing merged, d=0).
% >
% > XXX [/ IMPROTANT ] XXX
% >
% > A neuron is this decision making thing. We are finding the decision boundary, a representation of the weights w_0, w_1.
% >
% > Maximize the classification margin.
% >
% > Opposing data distribution (energy dissipation pathways)
% > fight for classification margin (compete for conduction resources)
% >
% > Spike encoding (optic nerve). Which spikes within the spike space are active.
% >
% > FF-RF (forward-float, reverse-float), that is the as close to non-destructive read as you get. Reading is adaptation, it changes the memory.
% >
% > We can go backwards, discretized by the spikes.
% >
% > Nature has a universal adaptive building block.
% >
% > Interacting collectives of this building block solves the problems that the brain solves.

% +--- Anti-Hebbian and Hebbian
% |
% V

% ref: ref: http://knowm.org/report-from-the-navy-karles-invitational-on-neuro-electronics/
%
% Where each operation results in Anti-Hebbian or Hebbian learning. At the lowest level, Anti-Hebbian just means “move the synapse toward zero” and Hebbian means “move it away from zero”.

% ref: http://knowm.org/knowm/
%
% “Anti-Hebbian and Hebbian” in honour of Donald O. Hebb
