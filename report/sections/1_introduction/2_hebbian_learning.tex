\subsection{Hebbian Learning}

What underlying principle or set of principles govern learning and the formation of memories? Inspired by recent discoveries in neuroscience and synaptic research, Donald Hebb proposed an iconic postulate in 1949, \textit{``When an axon of cell A is near enough to excite a cell B and repeatedly or persistently takes part in firing it, some growth process or metabolic change takes place in one or both cells such that A's efficiency, as one of the cells firing B, is increased.''} which is often summarized as \textit{``neurons which fire together, wire together''} \cite{hebbs_rule}. Ever since its formulation, Hebb's rule has been a cornerstone in understanding the inner workings of learning, memory formation and associative memory. Later research has extended Hebb's rule to include a notion of decay, where synaptic connections are weakened (and subsequently memories forgotten). Without this addition, all synapses would eventually converge to their maximum efficiency through synaptic potentiation, thus making them in-differentiable and thereby removing the selective property which is key for memory recall \cite{anti_hebbian}.
