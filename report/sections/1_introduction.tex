\section{Introduction}

Associative memory may be defined as \textit{``the ability to correlate different memories to the same fact or event''} \cite{memristor_conditioning}. Two broad categories of associative memory distinguish between memory recalled from partial information or cues (auto-associative memory), and memory recalled from related categories or concepts (hetero-associative memory). To give an example, auto-associative memory is used when asked to fill in the missing parts (e.g. \textit{``Which country in Europe starting with an ``F'' is known as ``the land of a thousand lakes''?''}), while hetero-associative memory is used when asked what thoughts comes to mind when presented with a given concept (e.g. \textit{``If I say elephant, you may say pink, big or animal.''}).

% === [ Subsections ] ==========================================================

\subsection{Hebbian Learning}

What underlying principle or set of principles govern learning and the formation of memories? Inspired by recent discoveries in neuroscience and synaptic research, Donald Hebb proposed an iconic postulate in 1949, \textit{``When an axon of cell A is near enough to excite a cell B and repeatedly or persistently takes part in firing it, some growth process or metabolic change takes place in one or both cells such that A's efficiency, as one of the cells firing B, is increased.''} which is often summarized as \textit{``neurons which fire together, wire together''} \cite{hebbs_rule}. Ever since its formulation, Hebb's rule has been a cornerstone in understanding how learning, memory formation and associative memory work. Later work has extended Hebb's rule to include a notion of decay, where synaptic connections are weakened (and subsequently memories forgotten). Without this addition, all synapses would eventually converge to their maximum efficiency through synaptic potentiation, thus making them indifferentiable and thereby removing the selective property which is key for memory formation \cite{anti_hebbian}.

% TODO: Add In other words, ... after \cite{hebbs_rule}.
% TODO: Rephrase the last paragraph; "... which key for memory formation."


% TODO: Add purpose, aim, and scope of the paper.
%\input{sections/1_introduction/1_project_aim_and_objectives}
