\section{Future Potential}

% ref: Connectome.
%
% 100 000 000 000 neurons
%
% 10 000 more synaptic connections

% ref: https://www.youtube.com/watch?v=JqMpGrM5ECo (The Human Brain Project - Video Overview)
%
%     100 000 000 000 neurons
% 100 000 000 000 000 synapses

% === [ Subsections ] ==========================================================

\subsection{Intelligent Systems}

Intelligence defined by ability to make predictions, not behaviour \cite{intelligence_is_prediction}.

% ref: https://www.youtube.com/watch?v=R7HxFhVQVr4 ("Why AHaH computing?")
%
% > Traditional computing would go an inch while biology is able to circle the world.
%
% > Effective intelligence.

% ref: Tim Molter HiPeac Prague 2016 Memristor Keynote
%
% How do we reach the power efficiency of a brain. "Don't simulate a brain, build a brain"
%
% We need to build a brain where the distance between memory and computation is 0.

% ref: http://knowm.org/the-constraints-for-neuromorphic-computing-systems-have-suddenly-changed/
%
% In almost every machine learning algorithm or program, there lies an equation that looks something like this:
%
% w_{t+1}=w_{t}+\delta w
%
% In other words, the weight is nudged a bit. Sometimes it’s nudged in a positive direction. Sometimes it’s nudged in a negative direction. Sometimes it’s a big nudge. Sometimes it’s a small nudge. Most of machine learning boils down to understanding how to nudge weights around.
%
% The space of possible memristors is enormous and most of them don’t work exactly how we want them too. Some are too stochastic (random). Some do not switch quickly enough, or they switch too fast. Some are not bi-directional (meaning that we can’t controllably increase and decrease the resistance via the applied voltage direction). Many memristors are not incremental, meaning that their resistance can only be changed in large steps. Others are only incremental in one direction and will abruptly change in the other direction. Still others will stop working after only a few thousand switching events. Indeed, when one stops theorizing and goes out to find a suitable memristor–things get very difficult very quickly!
%
% `It now seems clear that a capacity to learn would be an integral feature of the core design of a system intended to attain general intelligence, not something something to be tacked on later as an extension or an afterthought`
% - Nick Bostrom, “Superintelligence: Paths, Dangers, Strategies”
%
% Artificial General Intelligence.

\subsection{Energy Efficiency}

Network in its true sense, \textit{the net is doing all the work} \cite{net_doing_all_the_work}.

\cite{ahah}

%Current computer architectures are designed around major bottlenecks, huge amounts of data has to be shuffled back and forth to perform computations. Reaching its limits; transistors now so small that they only allow a single electron to pass through (similar in size to the ion channels). At this scale, problems arise when transistors may allow electrons to pass through, when they shouldn't and wise versa; which leads to unpredictable behaviour (small bursts of ones when should be be all zero, and wise versa.)

% ref: Tim Molter HiPeac Prague 2016 Memristor Keynote
%
% Down to 50 Angstroms device size theoretically possible.


%The brain, parallel, error correcting, memory efficient. (not doing a lot of unnecessary data shuffling?)

% ref: From BrainScales to Human Brain Project Neuromorphic Computing Coming of Age
%
% Human Brain Project.
%
% 50 000 000 synapses, and about 200 000 neurons
%
% 10 000 faster than biological real-time.
%
% Performance.
%
% 10 femtojoule
%
% 1 joule (in super computers) 10^14
%
% Physical models (neuromorphic)
%
% 10 000 000 times more energy efficient than the state-of-the-art HPC (comparable model)
%
% 10 000 less energy efficient than biology.
%
% Including all the overheads.
%
% # Time Scales
%
% Causality detection   | 10^{-4} sec | 0.1 sec          | 10^{-8} sec |
% Synaptic plasticity   | 1 sec       | 1000 sec         | 10^{-4} sec |
% Learning              | 1 day       | 1000 day         | 10 sec      |
% Development           | 1 year      | 1000 year        | 3000 sec    |
%
% 12 orders of magnitude
%
% Evolution             | > millennia  | > 1000 millennia  | > months    |
%
% > 15 orders of magnitude

%# Applications
%
% Reverse engineer biological data

% ref: Schmuker, Michael, Thomas Pfeil, and Martin Paul Nawrot. "A neuromorphic network for generic multivariate data classification."
%
% TODO: Check :) HBP Roadmap for 2023
%
% Spikey (Heidelberg Lab), commercially available.
% 384 neurons
% 100 000 plastic synapses
% 10.000 - 100.000 real-time
% put and pray into Laptop with USB.
%
% Neuromorphic computing
% Consistent concept for non-von Neumann, non-Turing computer architecture.
%
% Watch the market for Resistor memory, once it is there they will use it for HBP, but using CMOS for now, similar to how von-Neumann used Vacuum tubes before the arrival of the transistor.

% ref: http://knowm.org/how-to-build-the-ex-machina-wetware/
%
% Once you understand what is going on, you can start to understand how to harness it. Think about it. If a bunch of particles will spontaneously organize themselves out of a colloidal suspension to dissipate energy then what happens when we control the energy? What happens if we make the maximal-energy-dissipation solution the solution to our problem? Will the particles self-organize to solve our problem? The answer, it turns out, is a resounding “yes”.
%
% Find a way to use nature as nature itself does. To not compute a brain, but to actually build one. To find a way for matter to self-organize on a chip to solve computational problems.
%
% Memory and processing merge, voltages and clock-rates drop and power efficiencies explode.
%
% AHaH Computing is about understanding how to build circuits that adapt or learn to solve your problems and, as a result, dissipate more energy ‘as a reward’. The path to maximal energy dissipation is the path that solves your problem, and the result is that Nature self-organizes to solve your problem.

% ref: http://knowm.org/the-gordon-panthana-dialog/
%
% It of course does not mean you can magically make a circuit that consumes zero energy. It means that calculation of very large numbers of interacting adaptive variables via the separation of memory and processing is overwhelming less efficient than building a very large interacting adaptive system directly. The example was meant to show just how significant this contribution can be, and why having an intrinsically adaptive element (the memristor) is exciting and useful.
%
% Specifically we use the physics of memristors to eliminate the power and time normally associated with synaptic integration and adaptation.
%
% AHaH Computing embraces hardware as part of the solution and designs from the bottom-up (hardware) and top-down (AHaH compatible algorithms). The result is a broadly commercially viable machine learning system that can scale to biological levels of efficiency.

% ref: http://knowm.org/report-from-the-navy-karles-invitational-on-neuro-electronics/
%
% Dr. Kwabena Boahen
%
% Comparing the scaling problem of CMOS transistors to traffic. He showed that as CMOS transistors are scaled down, and the number of electron ‘lanes’ get close to one, the devices exhibit extremely high magnitude noise due to individual electrons becoming trapped in local energy minima (pot holes) and blocking the traffic. In essence, when the transistors are large, any “stuck electrons” are easily bypassed and their contribution to the total current is minimal. However, when the devices approach the natural ‘lane width’ of an electron (about 2.7 nm if a remember correctly), a stuck electron could reduce the current by 50% or even shut down the whole device. The answer to these problems, Dr. Boahen believes (and I agree), lie in distributed fault-tolerant analogue architectures inspired by the brain.


% TODO: Add section about independent component analysis? The underpinning of how memories are stored (Temporally mapped by the hypocampus? Stored in the synapses connecting a set of concepts?)?

%\subsection{Independent Component Analysis}

% ref: http://knowm.org/report-from-the-navy-karles-invitational-on-neuro-electronics/
%
% Dr. Marcel Adam Just
%
% For example, when somebody thinks about an “orange”, for example, their brains activate representations for a color, and a shape, and a fruit, and other associations. By identifying these “thought components” and using machine learning, they can predict with high accuracy what object the person is thinking about. They have gone further with this, decomposing human emotion into a “basis set” of three components that they call “Valence” (is it good or bad), “Arousal” and “Social Interaction”.
%
% What really surprises me is that the exact same places of a brain are active for the same abstract concepts from one person to the next. That is, when I think of an “orange” and parts of my cortex light up corresponding to shape, color, texture, etc…the exact same places in your brain light up to represent the same abstract concepts! It would appear that while the human brain must learn these concepts, the location for where the learning occurs is genetically determined.
