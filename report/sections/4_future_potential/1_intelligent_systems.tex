\subsection{Intelligent Systems}

Intelligence defined by ability to make predictions, not behaviour \cite{intelligence_is_prediction}.

% ref: http://knowm.org/the-constraints-for-neuromorphic-computing-systems-have-suddenly-changed/
%
% In almost every machine learning algorithm or program, there lies an equation that looks something like this:
%
% w_{t+1}=w_{t}+\delta w
%
% In other words, the weight is nudged a bit. Sometimes it’s nudged in a positive direction. Sometimes it’s nudged in a negative direction. Sometimes it’s a big nudge. Sometimes it’s a small nudge. Most of machine learning boils down to understanding how to nudge weights around.
%
% The space of possible memristors is enormous and most of them don’t work exactly how we want them too. Some are too stochastic (random). Some do not switch quickly enough, or they switch too fast. Some are not bi-directional (meaning that we can’t controllably increase and decrease the resistance via the applied voltage direction). Many memristors are not incremental, meaning that their resistance can only be changed in large steps. Others are only incremental in one direction and will abruptly change in the other direction. Still others will stop working after only a few thousand switching events. Indeed, when one stops theorizing and goes out to find a suitable memristor–things get very difficult very quickly!
%
% `It now seems clear that a capacity to learn would be an integral feature of the core design of a system intended to attain general intelligence, not something something to be tacked on later as an extension or an afterthought`
% - Nick Bostrom, “Superintelligence: Paths, Dangers, Strategies”
%
% Artificial General Intelligence.
